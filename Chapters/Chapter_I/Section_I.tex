\documentclass[../main.tex]{subfiles}

\begin{document}

\section{Before you open the box}
Game theory was invented by mathematicians and started with the study of parlor games, where objectives and rules are clear. The goal was to compute \textit{solutions} that could predict the behavior of rational agents and also serve as recommendations on what people should do. Because such solutions apply to many other situations, the scope of game theory went beyond parlor games -- it just happens that these situations seem easier to analyze. Game theory was conceived as a general theory of strategic interactions between rational agents. This ambitious agenda quickly found applications in many fields of social sciences, considerably expanding its scope beyond the study of small-scale interactions. 

Modern economic thought incorporated a big portion of game theory, taking the field beyond the realm of applied mathematics. Social scientists understand game theory as the language of social interactions and use it to explain the logic behind many real-world phenomena such as cultural norms, communication, reputation building, and oligopolies. Many of these concepts are intrinsically non-mathematical and could be fully described in words. The only reason economists choose not to do so is that the mathematical language of game theory is more compact, precise, and straightforward.

Social scientists approach game theory like any other class of models of social interactions. Social interactions are so complex that it is impossible to encompass all the rules and objectives of the \say{players}. Because of the multifaceted nature of the problem at hand, social scientists must choose which dimensions are relevant. A game-theoretic model is necessarily a result of this choice. The analyst can, at best, include only what he thinks is relevant in the model and leave the other dimensions outside -- which we call \say{outside of the model}. There is no such thing as a 1:1 map.

I will approach game theory as a social scientist. Game theory is not a general theory of everything, but a set of tools and concepts from which one can choose, combine, and apply to model social interactions. Their solutions are not intended as recommendations on how people should behave or predictions on how people will act. It is a narrative resource. A language with which one can narrate specific phenomena and derive nontrivial lessons from a particular aspect of the logic behind this social interaction.

Because the language of game theory is mathematical, my approach will be formal; but concepts will be explained in words the best as I can. Sometimes I will be rather technical, but only when the results demand so. This is the case of the proof of the existence of the Nash Equilibrium in chapter 2. Since this result is too important to be ignored, I can't help but include it in the text, even though there is nothing economically insightful in the proof itself. Although I will try to detail some proofs, I will not try to achieve maximum generality. I will try to make the maths as simple as I can while being rigorous.
\end{document}
