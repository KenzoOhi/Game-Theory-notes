\documentclass[../../main.tex]{subfiles}

\begin{document}
\section{Game theory}
\label{GTheory_I}
Game theory studies strategic interactions between multiple decision-makers -- often called \textbf{players}. It usually assumes that the agents are rational -- they know what actions to choose -- and strategical -- they take into account other players' strategies. Every game-theoretical model must start with a description of a game. It specifies what \textbf{actions} are possible, what are the \textbf{payoffs}, the \textbf{order of play}, and the \textbf{information} the players have -- do they know the payoffs? Do they observe other players' actions? A complete model must also contain assumptions about how the game will be effectively played. More technically, a model must be accompanied by a \textbf{solution concept}.

Because there are multiple dimensions in which we distinguish games, it is appropriate to give them names. We say a game is \textbf{cooperative} if the unit of analysis -- the player -- is interpreted as a group of decision-makers. In contrast, we say a game is \textbf{noncooperative} if the player is interpreted as a unique individual. 

Likewise, one can distinguish games as \textbf{strategic} or \textbf{extensive}. A strategic game is one where players take actions simultaneously, whereas, in an extensive game, players decide their actions sequentially. In sequential games, it is often important to distinguish whether players have full information about previous actions. We say a game has \textbf{imperfect information} in the latter case and \textbf{perfect information} in the former.

Finally, there are many possible solution concepts for a game, each applying to a different context or class of games. I will talk about Nash equilibrium in the first part of these notes. This is the most important concept. It applies to strategic games but can also be used in extensive games with perfect information. Other concepts such as the subgame perfect Nash equilibrium refine the concept of Nash equilibrium, and others, like Bayesian Nash equilibrium, are important on their own. They are used in games with imperfect information.

I am going to organize these notes based on these distinctions. Most of this study will focus on noncooperative games. I will start with strategic games and then proceed to extensive games. I study perfect information first and then imperfect information.


\end{document}

