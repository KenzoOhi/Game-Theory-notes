\documentclass[../../main.tex]{subfiles}

\begin{document}
\section{Strategic behavior}
I would like to reemphasize the importance of strategic players. The fact that players must take other actions into account makes a substantial difference in the analysis and is a crucial ingredient in game theory. There is no game theory without the idea of strategic behavior. To understand this concept, contrast it with a model of competitive markets. In this model, each agent's action only depends on the actions of the others through some aggregate parameter like prices. No agent can change prices individually so they cannot affect each other's behavior. On the other hand, if markets were not competitive, in the sense that each player's decision affects market prices, actions would have to be strategic. A classical example is that of oligopolistic firms competing with each other for market share.

Strategic behavior is an instance of the problem of externalities. As we know from Coase's theorem, externalities can be \say{negotiated away} with well-defined property rights as long as there are no transaction costs. In many of the examples we will see in game theory, these rights are well-defined. Players fully utilize property rights on their bodies to make whatever actions are permissible and relevant. But those same rights represent transaction costs as, in some situations, individuals can't force other players to take some desired action. This particular type of transaction cost is called an agency cost. As we will see in some key examples, many inefficiencies arise because there are agency costs. 

Not only is game theory a useful tool to indicate inefficiencies that may arise in social interactions, but it can also provide tools to understand ways around such inefficiencies. One canonical example is the prisoner's dilemma. In rough words, this is a situation where players cannot coordinate on a mutually beneficial profile of actions. This type of situation arises precisely because there are agency costs. Each individual has no control over what the other player will do. However, as we advance to the study of repeated games, we can envision a different setting for this game. Suppose instead that both players can play this game repeatedly and they can observe each other's past actions. In this way, it is possible to achieve a desirable equilibrium by allowing players to punish deviating behavior.
\end{document}